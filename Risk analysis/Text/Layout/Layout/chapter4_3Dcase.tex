\chapter{Three-dimensional reservoir case}
- concept of 1D case in chapter 3 transferred to a 3D case of a structural geological model
- same again here: relatively simple model including some important typical geological and structural features and giving it an economic meaning
- alternating layers of sandstones and shales, so typical reservoir and seal formations
- structurally defined by a tilted dome-formation (folding along in the direction of two perpendicular axes) which is displaced by a standard normal fault
- these features are arranged in a way, that they result in the formation of a trap, defined by the anticlinal features and the fault, as well as enabled by the presence of a reservoir rock
- considering a petroleum production industry perspective, actors/ decision makers would be interested in the evaluation of value this potential reservoir
- "traditionally", this is represented by volumetric calculations or estimations of the recoverable reserves
- here: maximum volume of entrapment and subsequently, relative or hypothetical volumes of recoverable reserve can be calculated (for each model realization)
- (later: numbers from example case are taken, to achieve estimations of the NPV for a development project realization)
- looking at this structural model, not only the calculated volume can be taken into account, but also risk factors, such as fault permeabity seal safety (and capacity?), possible juxtapositions and leaks

- as in the 1D case, uncertainties are introduced for a number of parameters in the model, in a way that they affect the resulting maximum volume of the trap (and following also the recoverable reserves)
- again, a specific loss function is designed and applied on the results from numerous iterations of the constructed uncertain model, thus resembling the different value estimations/predictions or decisions different actors might make
- (to bring this Bayesian method of using loss functions closer to the traditional application of deterministic decision-making in decision-trees, the continuous estimation space of the reservoir value (volume) is combined with the assumption, that the estimated value is linked to a consequent investment into a determined project development size. thus, the continuous approach is used to make the decision in a deterministic decision-making space. ))

- later on,  the effect of adding information (in the form of likelihoods) is examined, particularly by looking at the changes in decision-making after applying this Bayesian updating and uncertainty has been reduced

- in the following, the construction and design of this model is described

	\section{Constructing the three-dimensional structural geological model}
	- base of the structural geological model is a 3Dcubic  space (block)  with 2000 m size in X, Y and Z x3
	- place layer positional points in three lines, resembling three hypothetical seismic lines as base information
	-  
	
	\section{Identifying trap structures in the geological model}