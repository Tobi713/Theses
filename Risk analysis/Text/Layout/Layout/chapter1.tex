\chapter{Introduction} \label{chap:intro}
Bayesian methods are an intuitive approach to inference, naturally inherent in human thinking patterns and closely tied to processes of decision-making \citep{berger2013stat, davidson2015, jaynes1986bayesian}. Individuals are constantly faced with situations in which a decision has to be made, but only incomplete information is available. Such a problem necessitates an approach based on plausible reasoning, one which is intuitively structured in five stages \citep{jaynes1986bayesian}:
\begin{enumerate}
	\item Identify uncertainties and attempt to consider all possibilities that might arise.
	\item Based on all the information and past experience available, evaluate how likely every possibility is.
	\item Assess the probable consequences of single possible actions.
	\item Based on the foregone steps, make a decision \citep{jaynes1986bayesian}.
\end{enumerate}
This concept is relatable to a vast variety of problems, ranging from casual every-day situations to complex scenarios in large-scale economic decision-making: As a private person, should I take an umbrella with me today? As a company, should we invest in the development and realization of a certain project? Following this process of plausible reasoning, the quality of a decision is to be measured based on the preceding state of knowledge and reasonable expectations, not on the subsequent actual consequences \citep{jaynes1986bayesian}. In other words: A decision is optimal, as long as it is the best action given the information available to the decision-maker before making the decision, no matter if actual loss was incurred afterwards.\\
Bayesian decision theory and the related concepts of expected loss and loss functions have found increasingly common application in several economic sectors and fields of research, such as medicine \citep{ashby2000evidence, ashby2006bayesian, moye2006statistical} and machine learning \citep{barber2012bayesian, theodoridis2015machine}. Probabilistic approaches to decision-making have also become prevalent in the sector of hydrocarbon exploration and production (SOURCE). However, the methods here a mainly limited on (p10-p90, decision trees).......(see BRATVOLD)???\\
In geosciences, Bayesian inference has prominently found use in the context of geophysical inversion problems (see \citet{tarantola1982inverse, mosegaard2002probabilistic} and \citet{sambridge2002monte}). Recently, it was transferred by \citet{delaVarga2016} to the field of structural geological modeling. This has been enabled by progressing developments regarding implicit geological modeling functions based on interpolation \citep{hillier2014three, mallet1992discrete, lajaunie1997foliation} and the possibility of fully automated model reconstruction in particular.  \citet{delaVarga2016} regarded geological modeling as a Bayesian inference problem by relating additional geological information to prior model parameters in the form of likelihood functions, linking them in a non-parametric Bayesian network. Using Markov chain Monte Carlo sampling to explore resulting probability spaces, they attained posterior model suites with reduced uncertainties \citep{delaVarga2016}\\
This work builds upon their concept, exploring the potential significance their findings might have in the context of decision-making. Bayesian decision-theory is to be included in the step of model evaluation. This is achieved by assigning an economic meaning to the structural model and designing a case-specific custom loss functions to find decisions which are optimal related to the state of knowledge and the preferences of actor's with different risk-affinities. More specifically, the models are designed to represent potential petroleum systems. Consequently, the development of algorithms for automatic hydrocarbon trap recognition and volume calculation represent a central part of this work.\\
The main hypothesis of this work is that Bayesian inference and resulting changes in uncertainties in a geological setting have a significant effect on related value estimation and decision-making. It is furthermore postulated that loss functions can be customized to appropriately represent preferences of actors in the hydrocarbon sector and moreover illustrate the nature of decisions such actor's might make depending on their individual attitudes towards risk and in the face of different types of uncertainties. Changes in their respective decisions are treated as a suitable measure to assess the effect of updating model parameters with new geological information.