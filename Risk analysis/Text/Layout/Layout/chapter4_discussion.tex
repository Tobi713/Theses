	\chapter{Discussion}\label{cha:discussion}
	Based on the recent findings by \citet{delaVarga2016}, showing that structural modeling can be viewed as a problem of Bayesian inference, the aim of this work was to extend this approach by considering practical applications and the utility geological modeling might have in an economic context. A sector in which structural geological modeling is of central importance and commonly used is hydrocarbon exploration. This field is characterized by the necessity to regularly make decisions in the face of high risks and potentially high rewards. As these decisions are often closely linked to geological modeling and the estimation of reservoir-related values, this work aimed to use this context to extend the Bayesian inference step in geological modeling by evaluating its results in terms of influence on decision making. It was initially hypothesized that Bayesian updating by using likelihoods and the resulting change in uncertainty in a model, should have significant effects on subsequent decision-making.\\
	To observe this, geological models were interpreted as potential hydrocarbon systems. Algorithms for structural trap recognition and calculation of values relevant to respective decision-making were developed. A custom loss function was designed to express the environment in which such values are to be estimated and to represent preferences of differently risk-affine decision makers. These methods were for a conceptual 1D geological model and subsequently for a 3D structural model. Despite the great leap in complexity, the results from both models are widely similar. According to these, conducting Bayesian inference in the context of structual geological modeling of hydrocarbon system induced several main effects on reservoir value probability distributions and the consequent estimation step, i.e. the decision-making.\\
	Changes in the posterior score and $ROV$ distributions were characterized by shifting of probabilities, as well as raising and narrowing of modes. The latter effect can be interpreted as a translation of uncertainty reduction in the model, to uncertainty reduction in the resulting distribution. Lateral shifting can be related to modes of the distribution as well as the distribution as a whole. ...\\
	One major observation to be named is that it seem to be of central importance "where" in the model uncertainty is reduced, i.e. in which spatial area or regarding which model parameter. This appears to be of particular importance considering threshold values that lead to an abrupt cut-off and duality in the decision-making. In both types of model construction, 1D and 3D, this was directly related to sealing reliability. Thresholds regarding seal thickness (1D model) and Shale Smear Factor ($SSF_c$ in 3D model) had been defined in a way that introduced a significant possibility of complete failure of a trap. Consequently, it was observed that reducing the uncertainty in a way that narrowed the probability of a threshold-related parameter around its cut-off value, led to an amplification of large-scale bimodality of the score or $ROV$ probability distribution. Resulting Bayes action were characterized by an increased divergence, i.e. disagreement between the estimates of different decision-makers. Thus, it appears suitable to introduce a differentiation between model uncertainty and decision uncertainty. Reducing model uncertainty in the "wrong" areas seems to simply lead to a transformation of this uncertainty in the realm of decision-making. Overall uncertainty seems to be conserved, as the duality of the decision problem is amplified in the form of a more stretched-out bimodal distribution and diverging Bayes actions. According to this, it should be a foremost importance for each actor, to reduce the uncertainty regarding threshold-related factors which might decide between "success" and complete failure of a project.\\
	It furthermore can be argued that the degree of convergence of Bayes actions is a measure for the state of knowledge during the decision-making process.  
	
	3: high risk related to seal-reliability --> main factor for decision making: important to look for right "area" to reduce uncertainty
	
	%Comparing the 1D and the 3D geological model, there is a large difference in complexity behind the model construction, as well as the way in which the decisive posterior value distributions (scores and recoverable volumes, respectively) are attained. Nevertheless, results of evaluating these models through application of the custom loss function are widely similar in principle. The reference distributions from the prior-only Monte Carlo uncertainty propagation sampling are comparable between the two cases. They are generally characterized by broad (approximately normal) distributions with striking peaks at zero or negative values. Thus, the variety of possible value realizations depending on the prior parameter distributions, but also the cut-off threshold dependent cases of seal failures are represented. It was observed that Bayesian inference has to main effects on the characteristics of these value distributions, and consequently also on the realization of decisions from application of the custom loss function: (1) lateral shifting of the probability distributions and the Bayes actions; (2) narrowing of the distributions and convergence of the decisions of different risk-affine actors. 
	%As these two mechanism seem to act independently from one another, at least to a certain degree, the following conclusions can be derived:
	%\begin{enumerate}
	%\item \textbf{More information does not necessarily lead to better decisions.} In cases, in which no considerable reduction of uncertainty is achieved by Bayesian inference, but posterior distributions are primarily shifted, the range of different actor's decisions will shift accordingly, but not converge, while expected losses remain approximately constant. This was observed in particular for cases, in which the standard deviations provided in the likelihoods were relatively large, or uncertainty was reduced very proximal to a seal failure-defining threshold value, so that a relatively narrow distribution still resulted in an approximately 50-50 binary probability of success versus failure (see 1D case: reduction of uncertainty for a seal formation close to the threshold of 20~m).
	%\item \text{The higher the uncertainty reduction, the better the decisions,} i.e. it is the nature of the information
	%\end{enumerate}  
	
	1D CASE:
	
				-abstract case: easy model construction and straightforward design of a loss function making basic assumptions and taking relative values that can simply be exchanged
			
				-so representative in a "relative" way, mostly appropriate to illustrate principles and benefits of the methodology
				
				-decision making/ estimation is defined by the design of the loss function which includes framework parameters which depend on the problem environment (e.g. market, technical constraints, etc.)
			
				-similar actors but different behaviors concerning risk: different decisions but same general loss function path (just different minima of EL)
			
				-additional information changes decision (BA) and EL
			
				-magnitude of change varies not only with the nature of the information (magnitude of uncertainty reduction) but also with the risk parameter
			
				-some actors might benefit more from some specific additional information than others
			
				-quantifiable value of information??
				
				
				
				
				- The GREATER the REDUCTION IN UNCERTAINTY, the LOWER the relevance of the risk-factor and difference in actor's preferences
				
				- accurate decision-making vs. gaining of information that yields encouraging results (vs "disappointing" results)
				
				
				
	WHERE in the MODEL uncertainty is reduced is SIGNIFICANT for the DECISION MAKING PERSPECTIVE:
	uncertainty might be reduced greatly in a model, e.g. looking at IE,
	but this might happen to lead to high probabilities at areas which are critical to the decision-making problem, such as areas at which a threshold has been defined, leading to a stark cut-off limit between YES and NO (two extrema of decisions)
	==> so maybe, while the objective model-related uncertainty has been reduced, the DECISION UNCERTAINTY remains constant or might even rise??