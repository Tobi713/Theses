	\chapter{Discussion}\label{cha:discussion}
	Comparing the 1D and the 3D geological model, there is a large difference in complexity behind the model construction, as well as the way in which the decisive posterior value distributions (scores and recoverable volumes, respectively) are attained. Nevertheless, results of evaluating these models through application of the custom loss function are widely similar in principle. The reference distributions from the prior-only Monte Carlo uncertainty propagation sampling are comparable between the two cases. They are generally characterized by broad (approximately normal) distributions with striking peaks at zero or negative values. Thus, the variety of possible value realizations depending on the prior parameter distributions, but also the cut-off threshold dependent cases of seal failures are represented. It was observed that Bayesian inference has to main effects on the characteristics of these value distributions, and consequently also on the realization of decisions from application of the custom loss function: (1) lateral shifting of the probability distributions and the Bayes actions; (2) narrowing of the distributions and convergence of the decisions of different risk-affine actors. 
	As these two mechanism seem to act independently from one another, at least to a certain degree, the following conclusions can be derived:
	\begin{enumerate}
	\item \textbf{More information does not necessarily lead to better decisions.} In cases, in which no considerable reduction of uncertainty is achieved by Bayesian inference, but posterior distributions are primarily shifted, the range of different actor's decisions will shift accordingly, but not converge, while expected losses remain approximately constant. This was observed in particular for cases, in which the standard deviations provided in the likelihoods were relatively large, or uncertainty was reduced very proximal to a seal failure-defining threshold value, so that a relatively narrow distribution still resulted in an approximately 50-50 binary probability of success versus failure (see 1D case: reduction of uncertainty for a seal formation close to the threshold of 20~m).
	\item \text{The higher the uncertainty reduction, the better the decisions,} i.e. it is the nature of the information
	\end{enumerate}  
	
	1D CASE:
	
				-abstract case: easy model construction and straightforward design of a loss function making basic assumptions and taking relative values that can simply be exchanged
			
				-so representative in a "relative" way, mostly appropriate to illustrate principles and benefits of the methodology
				
				-decision making/ estimation is defined by the design of the loss function which includes framework parameters which depend on the problem environment (e.g. market, technical constraints, etc.)
			
				-similar actors but different behaviors concerning risk: different decisions but same general loss function path (just different minima of EL)
			
				-additional information changes decision (BA) and EL
			
				-magnitude of change varies not only with the nature of the information (magnitude of uncertainty reduction) but also with the risk parameter
			
				-some actors might benefit more from some specific additional information than others
			
				-quantifiable value of information??
				
				
				
				
				- The GREATER the REDUCTION IN UNCERTAINTY, the LOWER the relevance of the risk-factor and difference in actor's preferences
				
				- accurate decision-making vs. gaining of information that yields encouraging results (vs "disappointing" results)