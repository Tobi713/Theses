	\chapter{Discussion}\label{cha:discussion}
	Building upon the recent findings by \citet{delaVarga2016}, showing that structural modeling can be viewed as a problem of Bayesian inference, the aim of this work was to extend this approach by considering practical applications and the utility geological modeling might have in an economic context. A sector in which structural geological modeling is of central importance and commonly used is hydrocarbon exploration. This field is characterized by the necessity to make decisions in the face of high risks and potentially high rewards. As these decisions are often closely linked to geological modeling and the estimation of reservoir-related values, this is work aimed to use this context to extend the Bayesian inference step in geological modeling by evaluating its results in terms of influence on decision-making. It was initially hypothesized that Bayesian updating by using likelihoods and the resulting change of uncertainty in a model, should have significant effects on subsequent decision-making.\\
	To analyze this, geological models were interpreted as potential hydrocarbon systems. Algorithms for structural trap recognition and calculation of values relevant to respective decision-making were developed. A custom loss function was designed to express the environment in which such values are to be estimated and to represent preferences of differently risk-affine decision-makers. These methods were applied for a conceptual 1D geological model and subsequently for a 3D structural model. Despite the great leap in complexity, the results from both models are widely similar. According to these, conducting Bayesian inference in the context of structural geological modeling of hydrocarbon system induces several main effects on reservoir value probability distributions and the consequent estimation step, i.e. the decision-making.\\
	As score and $ROV$ probability distributions were defined to depend on model parameters directly, but also on their values related to other parameters and specific thresholds, changing the model uncertainty naturally resulted in a change of shape of related posterior distributions. For both model designs, the prior value probability distributions were characterized by a distinct bimodality formed by widely separated modes of opposing values (positive vs. negative or zero). By introducing likelihood functions, these modes were most often shifted or altered in their shape. The latter was observed as either an amplification of the mode, by raising its mean probability, while narrowing the range of probabilities (i.e. reducing the standard deviation), or the opposite, a diminishment of the mode up to total erasure of its probability. These changes were often accompanied by lateral shifting of a mode. It can be argued that these processes reflect a translation of the model uncertainty into uncertainty in the reservoir value distribution. Within this it can be differentiated between uncertainty related to single modes, and uncertainty regarding the overall distribution. In some inference cases, the degree of bimodality was increased. This was caused by amplificaton of one of the modes, while the other remained constant or was also raised. So while uncertainty was reduced for one or both modes, the overall spread of probabilities and thus the total uncertainty was conserved, indicated by the mean of the whole distribution located in between both modes. In other cases, one mode was diminished or even completely erased, while the other was amplified. Resulting in overall unimodal distributions with centralized means, characterized accordingly by a reduction in total uncertainty. As the process of decision-making is based on these value distributions by applying the custom loss function respectively, this overall uncertainty might well be referred to as "decision uncertainty", here.\\
	The decision-making of differently risk-affine actors, i.e. the position and spread of Bayes actions, changed relative to the properties of value probability distributions and their inherent decision uncertainty. Distinct bimodality between two extremes, i.e. high decision uncertainty, resulted in a higher spread of Bayes estimators. Reduction of the distribution to one mode conversely led to the convergence of different Bayes actions. Reduction of the overall decision uncertainty furthermore was accompanied by a decrease in expected loss for each decision. Consequently, it can be argued that the degree of convergence of Bayes actions and their expected loss can be considered measures for the state of knowledge during the decision-making process. The better this is, the more similar the decisions of differently risk-affine actors are. It can be assumed that given perfect information, all actors would bid on the same estimate, the true value, and expect no loss. The relevance of the risk-factor consequently decreases with higher reduction of decision uncertainty.\\
	One major observation to be named is that it seem to be of central importance "where" in the model uncertainty is reduced, i.e. in which spatial area or regarding which model parameter. This appears to be of particular significance considering threshold values that lead to an abrupt cut-off between two extrema of decision options. In both types of model construction, 1D and 3D, this is directly related to sealing reliability. Thresholds regarding seal thickness (1D model) and Shale Smear Factor ($SSF_c$ in 3D model) were defined in a way that introduced a significant possibility of complete failure of a trap. Consequently, it was observed that reducing the uncertainty in a way that narrowed the probability of a threshold-related parameter around its cut-off value, led to an amplification of the respective mode (negative or zero values) and thus emphasized the risk of complete failure. Resulting Bayes actions were characterized by an increase in divergence and expected losses. Reducing model uncertainty in the "wrong" areas seems to simply lead to a transformation of this uncertainty in the realm of decision-making. Overall uncertainty seems to be conserved or even increased (see 3D posterior model V), as the duality of the decision problem is amplified in the form of a more stretched-out bimodal distribution and diverging Bayes actions. According to this, it should be of foremost importance for each actor, to reduce the uncertainty regarding threshold-related factors which might decide between "success" and complete failure of a project. Some type of information might improve the potential for a positive outcome, but maintain the risk of failure. For making better decisions, elimination of such high risks should be a priority. In the context of hydrocarbon traps, this is greatly influenced by seal reliability.\\
	
	- IE not adequate method for finding this "where", as such a threshold value might dependent on more on the relation of different parameters, than on the realization of one in the model
	
	It has too be emphasized that the models constructed in this work were purely artificial and not based on real data. However, the 3D model in particular was designed to include some typical structural characteristics related to hydrocarbon systems and algorithms were developed to consider the most common conditions that define structural traps. The fact that similar observations were made for the 1D, as well as the 3D model, indicates that these might be appropriately representative to conceptually visualize the effects of the methods.\\
	For the purpose of a conceptual application, the customization of the loss functions was kept relatively simple. In fact, as it incorporated only a few basic assumptions, it might thus be more generally representative than a complex function which takes into account too specific aspects. The consideration of more details would have furthermore required extensive speculation, which would have presumably impaired the generalization potential of respective observations.
	
	%Comparing the 1D and the 3D geological model, there is a large difference in complexity behind the model construction, as well as the way in which the decisive posterior value distributions (scores and recoverable volumes, respectively) are attained. Nevertheless, results of evaluating these models through application of the custom loss function are widely similar in principle. The reference distributions from the prior-only Monte Carlo uncertainty propagation sampling are comparable between the two cases. They are generally characterized by broad (approximately normal) distributions with striking peaks at zero or negative values. Thus, the variety of possible value realizations depending on the prior parameter distributions, but also the cut-off threshold dependent cases of seal failures are represented. It was observed that Bayesian inference has to main effects on the characteristics of these value distributions, and consequently also on the realization of decisions from application of the custom loss function: (1) lateral shifting of the probability distributions and the Bayes actions; (2) narrowing of the distributions and convergence of the decisions of different risk-affine actors. 
	%As these two mechanism seem to act independently from one another, at least to a certain degree, the following conclusions can be derived:
	%\begin{enumerate}
	%\item \textbf{More information does not necessarily lead to better decisions.} In cases, in which no considerable reduction of uncertainty is achieved by Bayesian inference, but posterior distributions are primarily shifted, the range of different actor's decisions will shift accordingly, but not converge, while expected losses remain approximately constant. This was observed in particular for cases, in which the standard deviations provided in the likelihoods were relatively large, or uncertainty was reduced very proximal to a seal failure-defining threshold value, so that a relatively narrow distribution still resulted in an approximately 50-50 binary probability of success versus failure (see 1D case: reduction of uncertainty for a seal formation close to the threshold of 20~m).
	%\item \text{The higher the uncertainty reduction, the better the decisions,} i.e. it is the nature of the information
	%\end{enumerate}  